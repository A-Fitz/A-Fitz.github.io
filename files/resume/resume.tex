% Template taken from https://latexresu.me/ and modified by Austin FitzGerald
% (c) 2002 Matthew Boedicker <mboedick@mboedick.org> (original author) http://mboedick.org
% (c) 2003-2007 David J. Grant <davidgrant-at-gmail.com> http://www.davidgrant.ca
% (c) 2008 Nathaniel Johnston <nathaniel@nathanieljohnston.com> http://www.nathanieljohnston.com
%
% (c) 2012 Scott Clark <sc932@cornell.edu> cam.cornell.edu/~sc932
%
%This work is licensed under the Creative Commons Attribution-Noncommercial-Share Alike 2.5 License. To view a copy of this license, visit http://creativecommons.org/licenses/by-nc-sa/2.5/ or send a letter to Creative Commons, 543 Howard Street, 5th Floor, San Francisco, California, 94105, USA.

\documentclass[11pt]{article}
\newlength{\outerbordwidth}
\pagestyle{empty}
\raggedbottom
\raggedright
\usepackage[svgnames]{xcolor}
\usepackage{framed}
\usepackage{hyperref}
\usepackage{tocloft}
\usepackage{enumitem}
\usepackage{textcomp}
\usepackage[utf8]{inputenc}
\usepackage[T1]{fontenc}
\usepackage{fancyhdr}


%-----------------------------------------------------------
%Edit these values as you see fit

\setlength{\outerbordwidth}{2pt}  % Width of border outside of title bars
\definecolor{shadecolor}{gray}{0.75}  % Outer background color of title bars (0 = black, 1 = white)
\definecolor{shadecolorB}{gray}{0.93}  % Inner background color of title bars


%-----------------------------------------------------------
%Margin setup

\setlength{\evensidemargin}{-0.40in}
\setlength{\headheight}{0in}
\setlength{\headsep}{0in}
\setlength{\oddsidemargin}{-0.40in}
\setlength{\tabcolsep}{0in}
\setlength{\textheight}{9.5in}
\setlength{\textwidth}{7.2in}
\setlength{\topmargin}{-0.5in}
\setlength{\topskip}{0in}
\setlength{\voffset}{0.1in}


%-----------------------------------------------------------
%Custom commands
\newcommand{\resitem}[1]{\item #1 \vspace{-4pt}}
\newcommand{\resheading}[1]{
  \parbox{\textwidth}{\setlength{\FrameSep}{\outerbordwidth}
    \begin{shaded}
\setlength{\fboxsep}{0pt}\framebox[\textwidth][l]{\setlength{\fboxsep}{4pt}\fcolorbox{shadecolorB}{shadecolorB}{\textbf{\sffamily{\mbox{~}\makebox[6.962in][l]{\large #1} \vphantom{p\^{E}}}}}}
    \end{shaded}
  }\vspace{-11pt}
}
\newcommand{\ressubheading}[4]{
\begin{tabular*}{6.5in}{l@{\cftdotfill{\cftsecdotsep}\extracolsep{\fill}}r}
    \textbf{#1} & #2 \\
    \textit{#3} & \textit{#4} \\

\end{tabular*}\vspace{-6pt}}

\newcommand{\school}[3]{\vspace{1.5mm}
  \textbf{#1} \hfill \textit{#2} \hfill #3
}

\newcommand{\job}[3]{\vspace{1.5mm}
  \textbf{#1} \hfill #2 \linebreak \textit{#3}
}

\newcommand{\skill}[2]{
  \textbf{#1:} #2
}

\newcommand{\organization}[3]{\vspace{1.5mm}
  \textbf{#1} \hfill #2 \linebreak \textit{#3}
}

\newcommand{\project}[3]{\vspace{1.5mm}
  \textbf{#1} \hfill #2 #3
}

%-----------------------------------------------------------
\fancyhf{}
\renewcommand{\headrulewidth}{0.0pt}
\fancyfoot[L]{Last Updated 03/2024}
\pagestyle{fancy}

\begin{document}
    
\begin{tabular*}{7.2in}{l@{\extracolsep{\fill}}r}
	\textbf{\LARGE Austin FitzCorbett} & 
	\href{mailto:austin@fitzcorbett.com}{austin@fitzcorbett.com} - 
	\href{https://fitzcorbett.com/}{https://fitzcorbett.com} - 
	Aurora, IL
\end{tabular*}
\begin{tabular*}{7.2in}{l@{\extracolsep{\fill}}r}
	\textbf{\textit{né} Austin FitzGerald}
\end{tabular*}
%%%%%%%%%%%%%%%%%%%%%%%%%%%%%%
\resheading{Education}
%%%%%%%%%%%%%%%%%%%%%%%%%%%%%%
\begin{itemize}[leftmargin=*]
	
	\item[]
	      \school
	      {University of Wisconsin - Platteville}
	      {BS Software Engineering}
	      {09/2017 to 05/2021}
\end{itemize}
%%%%%%%%%%%%%%%%%%%%%%%%%%%%%%
\resheading{Skills}
%%%%%%%%%%%%%%%%%%%%%%%%%%%%%%
\begin{itemize}[leftmargin=*]
	\setlength\itemsep{0em}
	\item[] \skill{Programming Languages}{Java, C\#, TypeScript, JavaScript, Python}
	\item[] \skill{Markup Languages}{LaTeX, Markdown, HTML}
	\item[] \skill{Major Frameworks}{Angular, Spring, .NET}
	\item[] \skill{Databases}{SQL Server, PostgreSQL}
	\item[] \skill{Software}{Microsoft Office, Jira, BitBucket, Confluence, Bamboo, GitHub, GitLab, ServiceNow, WebSphere, AWS, SonarQube, Burp Suite, Redis, Docker}
         	\item[] \skill{Industry}{Agile, Scrum, Machine Learning, WCAG Accessibility, CI/CD, Application Security, Risk Assessment}
\end{itemize}
%%%%%%%%%%%%%%%%%%%%%%%%%%%%%%
\resheading{Work Experience}
%%%%%%%%%%%%%%%%%%%%%%%%%%%%%%
\begin{itemize}[leftmargin=*]
	\item[]
	      \job
	      {Nelnet (Remote)}
	      {04/2024 to Present}
	      {Software Architect}
               {\\Performed a retooling assessment (data center to cloud) for a major business line to the result of over \$1MM in annual savings, leading to a CBA being approved by executive leadership. Currently guiding a large development team through the first phase of retooling.\par
Software architect for two teams that support the distributed caching system and an enterprise service bus which all back our borrower and call center agent systems.\par
Previously, while supporting systems for IVR routing, call center dialing list generation, and SMS notifications supporting 16+ million customers, I navigated the development team through the later stages of a company-wide initiative that tripled the number of production tenant environments.}
	\item[]
	      \job
	      {Nelnet (Remote)}
	      {08/2022 to 04/2024}
	      {Software Architect - Application Security}
               {\\I worked with hundreds of software engineers and architects, managers, and cybersecurity analysts to improve the security posture of Nelnet systems by maintaining automated SAST and SCA implementations, leading threat modeling, and running our vulnerability management program. I regularly reviewed C\#, Java, TypeScript, and SQL for federal and commercial software systems that support 16+ million borrowers. I revived and spearheaded our successful Security Champions Program and architected its custom gamification platform, engaging 70 new security champions over the course of a year. Along with stakeholders from throughout the company, I designed security and compliance training and ensured compliance with contractual and regulatory requirements. I learned to quickly make decisions and assess ridk using the NIST SP 800 series and guidance from sources such as the NIST SSDF and the OWASP ASVS.}
	\item[]
	      \job
	      {Nelnet (Remote)}
	      {06/2021 to 08/2022}
	      {Software Engineer}
	      {\\I worked on a Scrum team that maintained over a dozen business critical software systems which supported internal and external stakeholders, including custom business continuity management software and a service that allowed Intuit Mint to collect federal student loan data on behalf of borrowers. I helped design, implement, and maintain five web applications (Spring/Angular/SQL Server) (<150k LoC each), leading the architecture of the last system we built. I also maintained software systems using technologies such as IBM Db2, AWS Lambda, the ServiceNow API, Helm, AngularJS, and WebSphere.\par
I helped improve our team's productivity by advocating for and implementing automated code quality
enforcement and procedures to streamline production deployments. I was a notable contributor to the Unifi Design System (https://unifi.nelnet.io/), where I engaged with UI/UX designers and accessibility engineers, then helped hire two developers to work on the project fulltime. I am also proud to have mentored an intern who was hired at Epic after graduation.}
	\item[]
	      \job
	      {UW-Platteville Residence Life (Platteville, WI)}
	      {01/2019 to 05/2021}
	      {Senior Assistant, Resident Assistant}
	      {\\As a senior assistant I was responsible for coordinating administrative duties, health and safety protocols, and acting as a co-supervisor for seven resident assistants. I juggled 30 hours of work per week along with a full load of classes, earning a 3.87 GPA, Senior Assistant of the Semester in the fall of 2020, and the Lifetime Residence Life Award in the spring of 2021.\par
As a resident assistant I was responsible for overseeing a community of approximately twenty-five undergraduate students, providing information on university resources, and developing an educational atmosphere. I was voted by my peers as the Staff Member of the Semester in the fall of 2019.}
	\item[]
	      \job
	      {Nelnet (Remote)}
	      {06/2020 to 08/2020}
	      {Software Engineering Intern}
	      {\\See "Software Engineer" above, a position that became a continuation of this internship.}
	\item[]
	      \job
	      {UW-Platteville CSSE Department (Platteville, WI)}
	      {09/2018 to 05/2020}
	      {Lab Assistant, Undergraduate Researcher}
	      {\\As a lab assistant I provided support to students enrolled in introductory computer science courses, assisting them understand and complete assessments related to core concepts regarding data structures, algorithms, and object-oriented programming.\par
As an undergraduate researcher I worked in a team under Dr. Selent to research, design, and develop multiple predictive models in Python used to predict student struggle in CSSE courses; our team won 2nd place in the 2019 WSTS Poster Symposium. I also worked under Dr. Alkhushayni to implement a unique set of chronic disease risk assessments for a personal health application.}
\end{itemize}
%%%%%%%%%%%%%%%%%%%%%%%%%%%%%%
\resheading{Other Experience}
%%%%%%%%%%%%%%%%%%%%%%%%%%%%%%
\begin{itemize}[leftmargin=*]
	\item[]
	      \organization
	      {Kane County Clerk}
	      {11/2022 to Present}
	      {Election Judge and Zone Manager}
	      {\\I served as an Election Judge in Kane County, Illinois during the 11/2022 and 05/2023 elections. I also served as a Zone Manager for multiple sites during the 03/2024 and 11/2024 elections.}
	\item[]
	      \organization
	      {Boys Baseball of Aurora}
	      {04/2024 to Present}
	      {Coach}
	      {\\Assistant coach for the 2024 summer 12/13U champions. Head coach for a fall 10/11U team.}
	\item[]
	      \organization
	      {Institute of Electrical and Electronics Engineers}
	      {06/2023 to Present}
	      {Member}
	      {\\IEEE Computer Society Member; Chicago Section; Member No. 99451128}
	\item[]
	      \organization
	      {Association for Computing Machinery}
	      {09/2018 to 05/2021}
	      {Student Member}
	      {\\I was a regular attendee of on-campus meetings and events such as the 2018, 2019, and 2021 International Collegiate Programming Contests.}
	\item[]
	      \organization
	      {Westview Elementary Coding Club}
	      {09/2018 to 03/2020}
	      {Volunteer Mentor}
	      {\\I mentored 3rd and 4th grade students as part of a coding club at Westview Elementary in Platteville, Wisconsin. We utilized curriculum provided by Code.org and Google\textquotesingle{}s CS First.}
\end{itemize}
%%%%%%%%%%%%%%%%%%%%%%%%%%%%%%
\resheading{Projects and Publications}
%%%%%%%%%%%%%%%%%%%%%%%%%%%%%%
\begin{itemize}[leftmargin=*]
	\item[]
	      \project
	      {\href{https://angular.unifi.nelnet.io/}{Nelnet Unifi Design System - Angular Library}}
	      {06/2021 to Present}
	      {\\I have made major contributions to the Angular library that implements Nelnet's "Unifi" Design System, which is used in dozens of customer-facing and internal-facing web applications including Nelnet's flagship loan servicing application used by 16 million borrowers (https://secure.nelnet.com/).}
	\item[]
	      \project
	      {\href{https://www.researchgate.net/publication/349682373_Health_tracker_data_acquisition_and_analysis_for_monitoring_health_trends_and_assessing_disease_risk}{HealthTracker: Monitoring Health Trends and Assessing Disease Risk}}
	      {08/2021}
	      {\\I completed this research under Dr. Alkhushayni as an undergraduate at UW-Platteville. This paper describes HealthTracker, a mobile health application to record, store, display, and analyze personal health data. This application allows an individual to log several types of data encompassing their personal health. HealthTracker serves as a model for both a recording and a recommending system.}
	\item[]
	      \project
	      {\href{https://afitz.space/files/2019_WSTS_Poster.pdf}{Applying Predictive Models to Course Curricula}}
	      {07/2019}
	      {\\I completed this research under Dr. Selent as an undergraduate at UW-Platteville. The purpose of our research was to provide data and tools necessary for students and faculty advisers to predict and prevent academic struggle. The dataset we used consists of historical grade data mined from UW-Platteville graduates and withdrawls between the years of 2013 and 2018. We used multiple predictive models to predict student success over a varying amount of time and compared the performance of these models.}
\end{itemize}
%%%%%%%%%%%%%%%%%%%%%%%%%%%%%%
\ 
\end{document}
